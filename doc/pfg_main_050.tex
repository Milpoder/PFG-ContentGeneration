\chapter[Conclusiones y trabajo futuro]{\label{identificadorReferenciaCruzada}
Conclusiones y trabajo futuro}


Los objetivos de este trabajo eran, la contrucción de una biblioteca para generación procedural de ciudades que fuese posible usar por cualquier persona, e importar las ciudades al motor de juegos Unity 3D para que sea posible usarlo en cualquier juego que se cree con este motor. Tras la realización de este trabajo, hemos llegado a la conclusión de que los algoritmos realizados son bastante competitivos con los que hay en el mercado, aunque el segundo algoritmo en ciudades muy grandes no se podría usar en cualquier juego, dado su coste. 

El trabajo futuro que se puede añadir a este proyecto sería por una parte la construcción definitiva de algoritmos genéticos ya que los algoritmos actuales son de una única iteracción, y comprobar con costes cual de las ciudades que van saliendo con la mezcla de genes ''edificios'', para obtener ciudades aún más competitivas, por otra parte este algoritmo se usará para la generación de varios juegos que se encuentran actualmente en desarrollo y que esperemos que el año que viene uno de ellos se encuentre en la plataforma de videojuegos ''STEAM'', para su venta y distribución. Todo el trabajo realizado se encuentra en GitHub en \cite{B13}.