%
% Frontmatter - Introducción. Los miembros del tribunal que juzgan los PFC's tienen muchas más memorias que leer, por lo que
%	agradecerán cualquier detalle que permita facilitarles la vida. En este sentido, realizar una pequeña introducción,
%	comentar la organización y estructura de la memoria y resumir brevemente cada capítulo puede ser una buena práctica
%	que permita al lector centrarse fácilmente en la parte que más le interesa.
%

\chapter[Introduccion]{
	Introduccion
}

Han pasado más de sensenta años desde que Alexander S. Douglas, crease el primer juego computerizado ''Nought and crosses'', a través del cual una persona y una máquina se veían enfrentadas en un tablero del popular juego de tres en raya. Hoy en día podríamos empezar a hablar que los videojuegos se han convertido en uno de los mercados económicos más valorados por la sociedad, prueba de este auge desde la segunda mitad del S. XX y comienzos del S. XXI han sido la creación de nuevas e importantes empresas que se encargan del desarrollo de este tipo de productos, ejemplo de ello pueden ser empresas  internacionales como ''ATARI'' o ''NINTENDO'', y nacionales como ''PYRO STUDIOS''.

Desde la incorporación de los videojuegos a la vida cotidiana, el ser humano a tratado de buscar no sólo el carácter lúdico de estos, sino también hallando la forma con la que poder trasmitir una serie de valores cívicos y conocimiento a la ciudadanía. Por ello se habla también de un segundo carácter que sería denominado como educativo. Por otro lado una de las características más llamativas del sector es su amplio repertorio para todos los públicos, idea inconcedible unas décadas atrás, ya que los videojuegos estaban ideados para un público en concreto, pero a raíz de los avances tecnológicos sufridos en esta industria, se ha producido una mayor difusión de los contenidos de los videojuegos, buscando que accesibles a todos el mundo. Por ende hoy en día, podemos encontrar videojuegos para niños, adolescentes, adultos y ancianos\cite{B12}.

En los últimos años se han realizado grandes avances en la computación grafíca, dando lugar a la realidad virtual y una gran cantidad de sensores que detectan el exterior para reproducirlo dentro del propio juego. Consolas como ''NINTENTO WII'' o ''XBOX ONE'' ha intentado enforcar sus últimos lanzamientos en cámaras que detecten el movimiento de la persona. Estos avances han permitido el desarrollo de habilidades diversas, como motoras o creativas. Por otro lado, empresas como ''OCULUS VR'' han creado gafas de realidad virtual con las que el usuario puede establecer mayor contacto con el mundo virtual.

Otro de los grandes avances que se ha producido en los videojuegos en la última década es la mejora de la inteligencia artificial en estos. Se han implementado algoritmos como el A*, para mejorar la búsqueda de caminos en bots, también hemos podido ver un gran avance en algoritmos de aprendizaje automático con los que la máquina, aprende de nuestras acciones y sabe como contrarestarlas, pero nosotros nos centraremos en la generación procedural.

Hay diez principales tipos de investigaciones en el campo de la inteligencia artificial en videojuegos, estos se dividen según tres perspectivas, la primera el uso de la inteligencia artificial en el cada área (métodos de perspectiva computacional), por otra parte podemos analizar la relación del área respecto a lo que hace el jugador, humano,(perspectiva del jugador) y para finalizar, podemos analizar la zona en la que el jugador realiza acciones (perspectiva de la interacción jugador-juego), el lugar dónde se desarrolla el juego.\cite{B1}


Las investigaciones son las siguientes:

\begin{center}

	\begin{itemize}
	
		\item Estudio de comportamiento de los NPC (Non-player character)\\
		\item Búsqueda y planificación\\
		\item Modelado de personajes\\
		\item Puntos de referencia en la inteligencia artificial de juegos\\
		\item Generación procedimental de contenido\\
		\item Narración computacional\\
		\item Creación de agentes creíbles\\
		\item Diseño de juego asistido con inteligencia artificial\\
		\item Inteligencia artificial general en juegos\\
		\item Inteligencia artificial en juegos comerciales\\
		
	\end{itemize}

\end{center}

Las áreas de investigación tienen relaciones bastante fuertes entre ellas, el avance de alguna de ellas implica que se pueden mejorar las investigaciones con las que tienen relaciones.


\subsection*{Objetivos del TFG}

\paragraph*{Primer objetivo:}
Consiste en la creación de una biblioteca que contenga algoritmos para la creación procedural de ciudades, cuya salida sea posible utilizarse en motores de juegos como ''UNITY 3D'' o ''UDK''. Esta librería tendrá licencia GPL para permitir que cualquier persona pueda modificarla y utilizarla libremente sin nigún problema.

\paragraph*{Segundo objetivo:}
Comparar los distintos algoritmos de generación procedural de ciudades de la biblioteca para ver cuál sería más óptimo para su utilización en un juego determinado, analizando los algoritmos a posteriori y obteniendo los superiores de las funciones.

\paragraph*{Tercer Objetivo:}
Facilitar la utilización de la biblioteca a cualquier persona que no entienda de programación ni de desarrollo de software en general, ya que se desea que llegue a la mayor cantidad de usuarios posible.

%
% SECCION
%
\subsection*{Estructura de la memoria}

\paragraph*{Estado del arte:}
En este capítulo se muestra información sobre las distintas áreas de investigación y las interacción entre estas, esta información ha sido extraída de \cite{B1}

\paragraph*{Generación procedural de contenidos:}
En este capítulo se habla de los distintos tipos de generación procedural, los algoritmos usados para esta y la representación de los distintos espacios de búsqueda además de las funciones de evaluación, esta información ha sido sacada de \cite{B1}\cite{B2}\cite{B3}\cite{B4}\cite{B5}\cite{B6}\cite{B7}\cite{B8}.

\paragraph*{Motores de Juego:}
En este capítulo se habla sobre lo que es un motor de juego y sus componentes, además de explicar como se elige este y se hace una elección para el proyecto, esta información ha sido sacada de \cite{B9}\cite{B10}\cite{B11}.
