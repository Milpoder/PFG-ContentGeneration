%
% Frontmatter - Introducción. Los miembros del tribunal que juzgan los PFC's tienen muchas más memorias que leer, por lo que
%	agradecerán cualquier detalle que permita facilitarles la vida. En este sentido, realizar una pequeña introducción,
%	comentar la organización y estructura de la memoria y resumir brevemente cada capítulo puede ser una buena práctica
%	que permita al lector centrarse fácilmente en la parte que más le interesa.
%

\chapter[Introduccion]{
	Introduccion
}

Hay diez principales tipos de investigaciones en el campo de la inteligencia artificial en videojuegos, estos se dividen según tres perspectivas, la primera el uso de la inteligencia artificial en el cada área (métodos de perspectiva computacional), por otra parte podemos analizar la relación del área respecto a lo que hace el jugador, humano,(perspectiva del jugador) y para finalizar, podemos analizar la zona en la que el jugador realiza acciones (perspectiva de la interacción jugador-juego), el lugar dónde se desarrolla el juego.\\


Las investigaciones son las siguientes:\\

\begin{center}

	\begin{itemize}
	
		\item Estudio de comportamiento de los NPC (Non-player character)\\
		\item Búsqueda y planificación\\
		\item Modelado de personajes\\
		\item Puntos de referencia en la inteligencia artificial de juegos\\
		\item Generación procedimental de contenido\\
		\item Narración computacional\\
		\item Creación de agentes creíbles\\
		\item Diseño de juego asistido con inteligencia artificial\\
		\item Inteligencia artificial general en juegos\\
		\item Inteligencia artificial en juegos comerciales\\
		
	\end{itemize}

\end{center}

Las áreas de investigación tienen relaciones bastante fuertes entre ellas, el avance de alguna de ellas implica que se pueden mejorar las investigaciones con las que tienen relaciones.\\


%
% SECCION
%
\subsection*{Estructura de la memoria}

\paragraph*{Investigaciones.}
Resumen capítulo

\paragraph*{Generación procedural de contenidos.}
Resumen capítulo
