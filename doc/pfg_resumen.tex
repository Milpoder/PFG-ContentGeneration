%
% Resumen del proyecto de fin de carrera
%

\section*{Resumen:}

Los videojuegos es uno de los sectores emergentes en los últimos años a nivel mundial. El proyecto está basado en la generación procedural de contenidos para videojuegos, en este caso, los contenidos que se van a generar son ciudades, para ello se ha creado una biblioteca de generación de ciudades con tres algoritmos, la biblioteca ha sido creada para Autodesk 3D Maya, ya que se puede trasladar de una manera muy sencilla los resultados a Unity 3D uno de los motores de videojuegos más usado en los últimos años. Dos de los algoritmos son bastante competentes con los que hay actualmente en el mercado y otro, es un algoritmo que necesitaría grandes requisitos para su uso, pero da mejores resultados que los anteriores. Tras la creación de los algoritmos se hizo un análisis a posteriori, llegando a la conclusión de que todos los algoritmos tienen una eficiencia O$(n)$, también se hizo un estudio de la memoria llegando a la conclusión de que los dos algoritmos con mejores tiempos, también usaban pocoa memoria, por lo que se podrían generar ciudades muy grandes en cualquier juego. Por último esta biblioteca se utilizará para generar ciudades en videojuegos futuros y está libre en GitHub.